\newpage
\section{Vorbereitung}
In der Vorlesung wurden ARM-Assembler und Inline-Assembler für den ARM-GCC\\
besprochen. Lesen Sie hierzu für weitere Informationen die Webseite\\
\url{http://www.ethernut.de/en/documents/arm-inline-asm.html} und\\
\url{http://infocenter.arm.com/help/index.jsp?topic=/com.arm.doc.dui0553a/CIHJJEIH.html}\\
(Cortex-M4 Devices Generic User Guide, Kapitel 3 $"$The Cortex-M4 Instruction Set$"$ im Ilias).
\section{Aufgabe 1}
Ergänzen Sie das Listing 6.1 mit Assembler-Befehlen in den angegebenen Bereich, so dass Ihr ergänzter Programmcode die Nummer \textit{number} verändert. Die Nummer muss bei jedem Durchlauf verdoppelt werden, bis sie den Wert 128 erreicht hat. Danach muss die Nummer wieder auf den Wert 1 gesetzt werden und die Verdoppelung vorne beginnen.\\
\definecolor{mygreen}{rgb}{0,0.6,0}
\definecolor{mymauve}{rgb}{0.58,0,0.82}
\lstset{
	breakatwhitespace=false,
	breaklines=true,
	commentstyle=\color{mygreen},
	frame=single,
	keepspaces=true,
	showstringspaces=false,
	keywordstyle=\color{blue},
	language=C,
	rulecolor=\color{black},
	stringstyle=\color{mymauve},
	tabsize=2
}
\lstinputlisting{../Aufgabe1/Aufgabe1.ino}
\begin{center}
	Listing 6.1: Rahmen für Aufgabe 1\\
\end{center}
\section{Aufgabe 2}
Ergänzen Sie das Listing 6.2 mit Assembler-Befehlen in den angegebenen Bereich, so dass das Array fibData mit den ersten dreizehn Fibonacci-Zahlen gefüllt wird. Dabei ist die dritte bis dreizehnte Fibonacci-Zahl jeweils aus ihren beiden vorhergehenden zu berechnen. Hinweis: Eine mögliche Wissenslücke bezüglich Fibonacci-Zahlen könnte Wikpedia füllen (\url{http://de.wikipedia.org/wiki/Fibonacci-Folge}). Beachten Sie, dass nicht alle Register benutzt werden können und verwenden Sie z.B. die Register R4 bis R7.\\
\lstinputlisting{../Aufgabe2/Aufgabe2.ino}
\begin{center}
	Listing 6.2: Rahmen für Aufgabe 2\\
\end{center}
Bestimmen Sie für den Assembler-Code beider Programme die Anzahl der Taktzyklen, die diese benötigen.\\ \\
Aufgabe 1:\\
cmp: 1 Taktzyklus\\
ite: 1 Taktzyklus oder 0 Taktzyklen (wenn am letzten Thumb-Befehl angehängt)\\
addne: 1 Taktzyklus\\
moveq: 1 Taktzyklus\\
Da addne nur dann ausgeführt wird wenn moveq nicht ausgeführt wird, hat Aufgabe 1 nur 3 bzw 2 (Wenn ite 0 Taktzyklen hat) Taktzyklen.\\ \\
Aufgabe 2:\\
ldr: 2 Taktzyklen\\
add.w 1 Taktzyklus\\ \\
back:\\
mov: 1 Taktzyklus\\
ldr: 2 Taktzyklen\\
cmp: 1 Taktzyklus\\
bhi: 1 Taktzyklus oder 1 + P Taktzyklen (wenn branch ausgeführt wird)\\
adds: 1 Taktzyklus\\
2 mal ldrb: 2 Taktzyklen\\
cmp: 1 Taktzyklus\\
add: 1 Taktzyklus\\
strb + ldr: 2 Taktzyklen\\
add: 1 Taktzyklus\\
str: 2 Taktzyklen\\
b: 1 + P Taktzyklen\\
---------------------------------\\
Sektion back = 1 + 2 + 1 + 1 + 1 + 2 + 1 + 1 + 2 + 1 + 2 + 1 + P = 16 + P\\
Zuerst wird ldr und add.w mit 2 + 1 = 3 Taktzyklen ausgeführt. Danach wird 11 mal die Sektion back mit 11 x (16 + P) = 176 + 11 x P Taktzyklen ausgeführt. Am ende wird mov, ldr, cmp und bhi mit 1 + 2 + 1 + 1 + P = 5 + P  Taktzyklen ausgeführt. Das ergibt 3 + 176 + 11 x P + 5 + P = 184 + 12 x P Taktzyklen.\\ \\
Überlegen Sie Sich weiterhin, wie Sie sicherstellen können, dass durch Ihren Programmcode veränderte Register nach der Durchführung ihre ursprüngliche Werte bekommen.\\ \\
Mithilfe des Ausdrucks $"$memory$"$ in der $"$clobber list$"$ wird dem Compiler mitgeteilt dass er alle Werte im Cache zwischenspeichern soll und nach der Ausführung des Assembler-Codes alle Werte neu laden soll.\\ \\

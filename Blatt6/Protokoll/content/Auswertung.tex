\newpage
\section{Vorbereitung}
In der Vorlesung wurden ARM-Assembler und Inline-Assembler für den ARM-GCC\\
besprochen. Lesen Sie hierzu für weitere Informationen die Webseite\\
\url{http://www.ethernut.de/en/documents/arm-inline-asm.html} und\\
\url{http://infocenter.arm.com/help/index.jsp?topic=/com.arm.doc.dui0553a/CIHJJEIH.html}\\
(Cortex-M4 Devices Generic User Guide, Kapitel 3 $"$The Cortex-M4 Instruction Set$"$ im Ilias).
\section{Aufgabe 1}
Ergänzen Sie das Listing 6.1 mit Assembler-Befehlen in den angegebenen Bereich, so dass Ihr ergänzter Programmcode die Nummer \textit{number} verändert. Die Nummer muss bei jedem Durchlauf verdoppelt werden, bis sie den Wert 128 erreicht hat. Danach muss die Nummer wieder auf den Wert 1 gesetzt werden und die Verdoppelung vorne beginnen.\\
\definecolor{mygreen}{rgb}{0,0.6,0}
\definecolor{mymauve}{rgb}{0.58,0,0.82}
\lstset{
	breakatwhitespace=false,
	breaklines=true,
	commentstyle=\color{mygreen},
	frame=single,
	keepspaces=true,
	keywordstyle=\color{blue},
	language=C,
	rulecolor=\color{black},
	stringstyle=\color{mymauve},
	tabsize=2
}
\lstinputlisting{../Aufgabe1/Aufgabe1.ino}
\begin{center}
	Listing 6.1: Rahmen für Aufgabe 1\\
\end{center}
\section{Aufgabe 2}
Ergänzen Sie das Listing 6.2 mit Assembler-Befehlen in den angegebenen Bereich, so dass das Array fibData mit den ersten dreizehn Fibonacci-Zahlen gefüllt wird. Dabei ist die dritte bis dreizehnte Fibonacci-Zahl jeweils aus ihren beiden vorhergehenden zu berechnen. Hinweis: Eine mögliche Wissenslücke bezüglich Fibonacci-Zahlen könnte Wikpedia füllen (\url{http://de.wikipedia.org/wiki/Fibonacci-Folge}). Beachten Sie, dass nicht alle Register benutzt werden können und verwenden Sie z.B. die Register R4 bis R7.\\
\lstinputlisting{../Aufgabe2/Aufgabe2.ino}
\begin{center}
	Listing 6.2: Rahmen für Aufgabe 2\\
\end{center}
Bestimmen Sie für den Assembler-Code beider Programme die Anzahl der Taktzyklen, die diese benötigen.\\ \\
Überlegen Sie Sich weiterhin, wie Sie sicherstellen können, dass durch Ihren Programmcode veränderte Register nach der Durchführung ihre ursprüngliche Werte bekommen.\\


\newpage
\section{Vorbereitung}
Infomaterial finden Sie u.a. im Ilias:
\begin{itemize}
	\item ARMv7-M Architecture Reference Manual, Kapitel A4 ($"$The ARMv7-M Instruction Set$"$) und A5 ($"$Thumb Instruction Set Encoding$"$)
	\item Cortex-M4 Devices Generic User Guide, Kapitel 3 $"$The Cortex-M4 Instruction Set$"$
\end{itemize}
\section{Aufgabe 1}
Sehen Sie sich an, wofür die Optionen \textit{-O0}, \textit{-O2} und \textit{-Os} des Compilers \textit{arm-none-eabi-gcc} verwendet werden, z.B. \url{https://gcc.gnu.org/onlinedocs/gcc-6.4.0/gcc/Optimize-Options.html}.\\ \\
\begin{itemize}
	\item -O0: Verringert die Kompilierungszeit und macht das beim debuggen die erwarteten Ergebnisse produziert werden.
	\item -O2: GCC führt nahezu alle unterstützen Optimierungen aus die nicht einen Speicher-Geschwindigkeit-Kompromiss eingehen. Diese Option erhöht die Kompilierungszeit und die Performance des generierten Code.
	\item -Os: Optimiert die Code-Grö\ss{}e.
\end{itemize}
Setzen Sie den Pfad für ausführbare Programme auf die Tools der (TivaC-)Toolchain der Energia-Entwicklungsumgebung und öffnen Sie ein Fenster mit einer Kommandozeile. Ermitteln Sie Ihre Comiler Version (\textit{arm-none-eabi-gcc --version}). Idealerweise haben Sie die Version 6.3.1.\\ \\
Der Pfad kann unter Systemsteuerung/System und Sicherheit/System/Erweiterte Systemeinstellungen/Erweitert/Umgebungsvariablen gesetzt werden.\\ \\
Laden Sie die vier Dateien \textit{blink.c}, \textit{startup\_gcc.c}, \textit{EK-TM4C123GXL.ccxml} und \textit{blink.ld} aus ILIAS. Für diese Aufgabe betrachten wir hauptsächlich die Datei blink.c (s. Listig).\\ \\
Übersetzen Sie die Dateien und bauen Sie eine ELF-Datei für das TivaC-Board mit Optimierung \textit{-O0}. Ermitteln Sie die Programmgrö\ss{}e und generieren Sie ein Disassembly. Sie können folgende Befehle in der Kommandozeile ausführen (oder sich ein Makefile schreiben). Für die folgenden Befehle ist der \textit{{PATH\_TO\_SYSTEM\_DIR}} durch den Pfad zum TivaC $"$system$"$-Ordner in ihrer Energia-Installation zu ersetzen, z.B. $"$C:\textbackslash Users\textbackslash User\textbackslash AppData\textbackslash Local\textbackslash Energia15\textbackslash packages\textbackslash energia\textbackslash hardware\textbackslash tivac\textbackslash 1.0.3\textbackslash system$"$:\\ \\
\textit{arm-none-eabi-gcc -mcpu=cortex-m4 -mthumb -O0 -I{PATH\_TO\_SYSTEM\_DIR} -std=gnu11 -c -o blink.o blink.c}\\ \\
\textit{arm-none-eabi-gcc -mcpu=cortex-m4 -mthumb -O0 -I{PATH\_TO\_SYSTEM\_DIR} -std=gnu11 -c -o startup\_gcc.o startup\_gcc.c}\\ \\
\textit{arm-none-eabi-gcc -mcpu=cortex-m4 -mthumb -O0 -T blink.ld -Xlinker --gc-sections -o blink.elf blink.o startup\_gcc.o}\\ \\
\textit{arm-none-eabi-size blink.elf}\\ \\
\textit{arm-none-eabi-objdump -h -S blink.elf > blink0.asm}
\definecolor{mygreen}{rgb}{0,0.6,0}
\definecolor{mymauve}{rgb}{0.58,0,0.82}
\lstset{
	breakatwhitespace=false,
	breaklines=true,
	commentstyle=\color{mygreen},
	frame=single,
	keepspaces=true,
	keywordstyle=\color{blue},
	language=C,
	rulecolor=\color{black},
	stringstyle=\color{mymauve},
	tabsize=2
}
\lstinputlisting{../Quellcode/blink.c}
\begin{center}
	Listing der Datei blink.c
\end{center}
Sie können das Programm mit den folgenden Schritten auch auf das Board überspielen:\\
\textit{DSLite load -c EK-TM4C123GXL.ccxml blink.elf}\\ \\
Sie sollten nun die Grö\ss{}e des Programmcodes ermittelt und u.a. die Datei \textit{blink0.asm} generiert haben. In der Datei \textit{blink0.asm} finden Sie ein Disassembly des Programms \textit{blink.c}.\\ \\
Todo: Grö\ss{}e des Programmcodes aufschreiben.\\ \\
Gehen Sie zu der 10. Zeile nach der Zeile mit \textit{000029c <main>}: Die Zeile hat die folgende Form (erste drei Zeichen ggf. anders):\\
\textbf{2ae:}~~ \textit{4b0d\quad \quad \quad ldr \quad r3, [pc, \#52]\quad \quad ; (2e4 <main+0x48>)}\\ \\
Erklären Sie, was Ihnen das Disassembly in den Spalten zeigt. Erklären Sie weiterhin genau die Anweisungen der Assemblerbefehle dieser und der folgenden 18 Zeilen bis zur Zeile mit\\
\textbf{2da:}~~ \textit{e7ee\quad \quad \quad b.n\quad 2ba <main+0x1e>\quad \quad }\\ \\
Todo: Spalten und Assemblerbefehle erklären.\\ \\
\section{Aufgabe 2}
Führen Sie nun die oben dargestellten Schritte mit der Option \textit{-O2} aus. Generieren Sie ein weiteres Disassembly in einer Datei \textit{blink2.asm}.\\ \\
Vergleichen Sie die beiden Grö\ss{}en der Programmcodes, die mit den unterschiedlichen Optimierungsparametern generiert wurden. Vergleichn Sie auch den Inhalt der Dateien \textit{blink0.asm} und \textit{blink2.asm}. Welche wesentliche Änderung hat der Compiler am auszufürenden Binärcode (siehe die Disassemblies \textit{.asm}) durchgeführt?\\ \\
Todo: Aufgabe 2 lösen.\\ \\
\section{Aufgabe 3}
Ändern Sie nun das Programm, indem Sie das Keyword \textit{volatile} im Quellcode \textit{blink.c} löschen. Führen Sie die oben beschriebenen Schritte mit der Option \textit{-O2} nochmals durch und sehen Sie Sich das neue Disassembly an.\\ \\
Was und warum hat der Compiler nun etwas Anderes generiert? Wie wirkt sich dies auf die Programmausführung auf dem Board aus?\\ \\
Todo: Aufgabe 3 lösen.\\
